\chapter{导数与微分}  
\section{导数的有关定义}
\begin{definition}{导数}{derivative}
    设函数 $y=f(x),x\in D$。现有一点 $x_0\in D$且 $x_0+\Delta x \in D$。定义 $\Delta y=f(x_0+\Delta x)-f(x_0)$。

    若极限
    \begin{align}
        \lim_{\Delta x \to 0}\frac{\Delta y}{\Delta x}
    \end{align}
    存在,则称 $f(x)$ 在 $x_0$ 处可导,称该极限为 $f(x)$ 在 $x_0$ 处的导数。记作
    \begin{align}
        \lim_{\Delta x \to 0}\frac{\Delta y}{\Delta x} \triangleq f'(x_0)
    \end{align}
\end{definition}

\begin{note}
    \begin{enumerate}
        \item 等价定义:
        
        由于当 $x \to x_0$时,有 $f(x) \to f(x_0)$,而 $\Delta x =x-x_0$,$\Delta y = f(x)-f(x_0)$,从而导数又可以表示为
        \begin{align}
            \lim_{x \to x_0}\frac{ f(x)-f(x_0)}{x-x_0}\triangleq f'(x_0)
        \end{align}
        \item 左右导数
        
        $\Delta x \to 0$ 分为两种情况:$\Delta x  \to 0^{-}$ 以及
        $\Delta x  \to 0^{+}$;同理,$ x \to a$也分为两种情况:$ x \to a^{-}$以及 $ x \to a^{+}$。从而我们可以类似地得到:

        \begin{align}
            \lim_{\Delta x \to 0^{-}}\frac{\Delta y}{\Delta x} \triangleq f_{-}'(x_0) 
       \end{align}

       \begin{align}
           \lim_{\Delta x \to 0^{+}}\frac{\Delta y}{\Delta x} \triangleq f_{+}'(x_0) 
       \end{align}

        并且,可以给出一点导数存在的充要条件,也即:

        $f'(x_0)$ 存在 $\Leftrightarrow f_{-}'(x_0),f_{+}'(x_0) $ 均存在且相等。

        \item 可导性与连续性
        
        $f(x)$ 在 $x_0$ 处可导,则 $f(x)$ 在$x_0$ 处连续,反之则不然。证明从略。
        \begin{example}
            设 \begin{align*}
                f(x) = \begin{cases}
                e^x-1, & x<0,\\
                \ln(1+2x), & x \ge 0
            \end{cases}
        \end{align*},
            
            求 $f'(0)$。
        \end{example}
        
        \begin{solution}
            先考虑连续性。
        
            由于 $f(0-0)=0=f(0+0)$,故 $f(x)$ 在 $x=0$ 处连续。
        
            再考虑可导性。
        
            由于 $f_{-}'(0)=\lim\limits_{x\to 0_{-}}\frac{f(x)-0}{x-0}=\lim\limits_{x \to 0_{-}}\frac{e^x-1}{x}=1$,
        
            $f_{+}'(0)=\lim\limits_{x\to 0_{+}}\frac{f(x)-0}{x-0}=\lim\limits_{x \to 0_{-}}\frac{\ln(1+2x)}{x}=2 \neq f_{-}'(0)$,
        
            故 $f'(0)$ 不存在。
        \end{solution}

        \item 已知 $f(x)$ 连续,
        
        若 $\lim\limits_{x \to a}\frac{f(x)-b}{x-a}=A$,则 $f(a)=b,f'(a)=A$。

        这是因为当 $\lim\limits_{x \to a}\frac{f(x)-b}{x-a}=A$时,由于分母为 $0$,则分子必须也为 $0$ 才能存在极限;根据导数定义又能推出 $f'(a)=A$。
        \end{enumerate}
        \end{note}
    \section{可微}
    \begin{definition}{可微}{kewei}
    设函数 $y=f(x),x\in D$。现有一点 $x_0\in D$且 $x_0+\Delta x \in D$。定义 $\Delta y=f(x_0+\Delta x)-f(x_0)$。若$\Delta y=A\Delta x+o(\Delta x)$,则称 $f(x)$在 $x_0$ 可微。将 $A\Delta x$记作 $\dd{y}$,称为 $f(x)$ 在 $x=x_0$处的微分。
    \end{definition}
    
    \begin{note}
        \begin{enumerate}
            \item 可导 $\Leftrightarrow$ 可微;
            \item 若$\Delta y = A\Delta x+o(\Delta x)$,则 $A=f'(x_0)$;
            \item 设 $y=f(x)$ 可导,则 $\dd{y} = \dd{f(x)}=f'(x)\dd{x}$,如 $\dd{x^3}=3x^2\dd{x}$。 
        \end{enumerate}
    \end{note}
    \section{求导工具}
    (一)求导公式:
    \begin{enumerate}
        \item $C'=0$
        \item $(x^a)'=ax^{a-1}$
        \item $(a^x)'=a^x\ln a$,特别地,$(e^x)'=e^x$
        \item $(\log_a x)'=\displaystyle\frac{1}{x\ln a}$,特别地,$(\ln x)'=\displaystyle\frac{1}{x}$
        \item $(\sin x)'=\cos x,(\cos x)'=-\sin x,(\tan x)'=\sec ^2 x,(\cot x)'= - \csc^2 x,$
        
        $(\sec x)'=\sec\tan x,(\csc x)' = -\csc\cot x$
        \item $(\arcsin x)' = \displaystyle\frac{1}{\sqrt{1-x^2}},(\arccos x)' = -\displaystyle\frac{1}{\sqrt{1-x^2}},(\arctan x)' = \displaystyle\frac{1}{1+x^2}$,
        
        $(\arccot x)' = -\displaystyle\frac{1}{1+x^2}$。
    \end{enumerate}

    (二)四则运算:
    \begin{enumerate}
        \item $(u\pm v)'=u'\pm v$
        \item 
        \begin{enumerate}
            \item $(ku)'=ku'$
            \item $(uv)'=uv'+u'v$
            \item $(uvw)'=u'vw+uv'w+uvw'$
            \begin{example}
                $f(x) = x(x+1) \cdots (x+99)$,求$f'(0)$。
            \end{example}
            \begin{solution}
                记$g(x) = (x+1)(x+2)\cdots(x+99)$,根据乘积求导法则,$(xg(x))' = g(x)+xg'(x)$,将$x=0$代入得 $f'(0)=99!$。
            \end{solution}
        \end{enumerate}
        \item $(\displaystyle\frac{u}{v})' = \displaystyle\frac{u'v-uv'}{v^2}$。
    \end{enumerate}

    (三) 复合函数
    \begin{theorem}{链式法则}{chain value}
        若 $y = f(u)$ 可导,$u = \varphi(x)$ 可导且 $\var\phi'(x)\neq 0$,则 $y=f(\varphi(x))$可导且
        \begin{align*}
            \frac{\dd{y}}{\dd{x}} &= \frac{\dd{y}}{\dd{u}}\cdot \frac{\dd{u}}{\dd{x}}\\
            &=f'(u) \cdot \varphi'(x)\\
            &=f'[\varphi(x)]\cdot \varphi'(x)
        \end{align*}
    \end{theorem}

    \begin{proof}
        由于 $\varphi'(x)=\lim\limits_{\Delta x \to 0}\displaystyle\frac{\Delta u}{\Delta x}\neq 0$,故 $\Delta u$ 与 $\Delta x$ 为同阶无穷小,也即 $\Delta u = O(\Delta x)$。而 
        \begin{align*}
        \displaystyle\frac{\dd{y}}{\dd{x}} &= \lim\limits_{\Delta x \to 0}\displaystyle\frac{\Delta y}{\Delta x}\\
        &=\lim\limits_{\Delta x \to 0}\displaystyle\frac{\Delta y}{\Delta u} \cdot \displaystyle\frac{\Delta u}{\Delta x}\\
        &=\lim\limits_{\Delta x \to 0}\displaystyle\frac{\Delta y}{\Delta u} \cdot \lim\limits_{\Delta x \to 0}\displaystyle\frac{\Delta u}{\Delta x}\\
        &=\lim\limits_{\Delta u \to 0}\displaystyle\frac{\Delta y}{\Delta u} \cdot \varphi'(x) \text{(注:这一步用到了同阶无穷小这个条件)}\\
        &=f'[\varphi(x)]\cdot\varphi'(x). 
        \end{align*}
        \qed
    \end{proof}

    \begin{example}
        设 $y = \ln\arctan x^2$,求 $\displaystyle\frac{\dd{y}}{\dd{x}}$。
    \end{example}

    \begin{solution}
        $\displaystyle\frac{\dd{y}}{\dd{x}} = \displaystyle\frac{1}{\arctan x^2}\cdot \displaystyle\frac{1}{1+(x^2)^2}\cdot 2x$
    \end{solution}

    \begin{example}
        $ y = e^{(\sin{\frac{1}{x}})^2}$,求 $y'$。
    \end{example}

    \begin{solution}
        $ y' = e^{(\sin{\frac{1}{x}})^2}\cdot 2\sin{\frac{1}{x}}\cdot \cos{\frac{1}{x}} \cdot (-\frac{1}{x^2})$
    \end{solution}

    (四)反函数导数
    
    $ y = f(x) \Rightarrow  x = \varphi(y)$
 





