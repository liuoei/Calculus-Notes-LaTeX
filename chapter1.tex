\chapter{极限与连续}
\section{极限的有关定义}
\begin{definition}{数列极限}{lim}
数列$\{a_n\}$,若$ \forall\varepsilon>0 $,$ \exists N>0 $,当$ n>N $时,有
\begin{align}
    \vert{a_n-A}\vert<\varepsilon
\end{align}
则称数列$\{a_n\}$的极限为$A$(或:收敛于$A$),记作
\begin{align}
    \lim_{n\to \infty} a_n=A 
\end{align}
\end{definition}



\begin{definition}{函数极限-1}{limf1}
函数$f(x)$,若$ \forall\varepsilon>0 $,$ \exists \delta>0 $,当$ 0<\vert x-a\vert <\delta $时,有
\begin{align}
    \vert{f(x)-A}\vert<\varepsilon
\end{align}
则称函数$f(x)$的极限为$A$,记作
\begin{align}
    \lim_{x\to a} f(x)=A_1 
\end{align}
\end{definition}

\begin{note}
\begin{enumerate}
    \item 若$x \to a$,则$x\neq a$.如:$ \lim\limits_{x\to 0}\displaystyle\frac{0}{x^3}=0 $;
    \item $\lim\limits_{x \to a}f(x)$ 与$ f(a) $ 无关。如:$\lim\limits_{x \to 1}\displaystyle\frac{x^2-1}{x-1}=\lim\limits_{x \to 1}(x+1)=2$;
    \item $x \to a$分为$ x \to a^{+} $和$ x \to a^{-}$;
    \item 我们称$0<\vert x-a\vert <\delta$为$ a $的去心邻域;
    \item $\lim\limits_{x \to a^-}\triangleq f(a-0)$(左极限);
    $\lim\limits_{x \to a^+}\triangleq f(a+0)$(右极限)。
\end{enumerate}
\end{note}

\FiveStar $\lim\limits_{x \to a} f(x)$存在$\iff f(a-0),f(a+0)$都存在且相等。

\begin{definition}{函数极限-2}{limf2}
函数$f(x)$,若对于$ \forall\varepsilon>0 $,$ \exists X>(<)0 $,当$ x>X(<-X) $时,有
\begin{align}
    \vert{f(x)-A}\vert<\varepsilon
\end{align}
则称函数$f(x)$的极限为$A$,记作
\begin{align}
    \lim_{x\to +\infty(-\infty)} f(x)=A 
\end{align}
\end{definition}

如,对于函数$f(x)=\arctan{x}$有:
 $\lim\limits_{x\to +\infty} f(x)=\displaystyle\frac{\pi}{2}$,$\lim\limits_{x\to -\infty} f(x)=-\displaystyle \frac{\pi}{2}$

 \begin{figure}[htbp]
    \centering
    \includegraphics[width=0.7\textwidth]{2.eps}
    \caption{$f(x)=\arctan{x}$的图像}
  \end{figure}
  
\begin{definition}{无穷小}{wqx}
若 $ \lim\limits_{x \to a} \alpha(x)=0$,则称$ \alpha(x)$当$x \to a$时为无穷小。
\end{definition}

\begin{note}
\begin{enumerate}
  \item 0是无穷小,但无穷小不一定为0;
  \item $\alpha(x)\neq 0$,$\alpha(x)$是否为无穷小与$ x $的\CJKunderdot{趋向}有关;如,$\alpha=3(x-1)^2$,而$\lim\limits_{x \to 1}\alpha=0 $,则$3(x-1)^2$当$x \to 1$时是无穷小。
  \item 设$\alpha \to 0,\beta \to 0$,有如下三种情形:
  
  (a) $\lim \displaystyle\frac{\beta}{\alpha}=0$,称$ \beta $为$ \alpha $的高阶无穷小,记作$\beta=o(\alpha)$;
  
  (b) $\lim \displaystyle\frac{\beta}{\alpha}=k(\neq \infty,0)$,称$ \beta $为$ \alpha $的同阶无穷小,记作$\beta=O(\alpha)$(特例:$\lim \displaystyle\frac{\beta}{\alpha}$=1,则称$ \beta $与$ \alpha $为等价无穷小,记作$\beta \sim \alpha$)。
\end{enumerate}
\end{note}


\section{极限的性质}
\subsection{极限的一般性质}
下面我们开始介绍极限的一般性质,并给出相关的证明。主要有:唯一性、保号性(重点)两个性质。
\begin{enumerate}
    \item 唯一性
    \begin{property}
        极限存在必唯一。
    \end{property}
    
    \begin{proof}
    设$\lim\limits_{x \to a}f(x)=A $
    又$ \lim\limits_{x \to a}f(x)=B $,并不妨设$A>B$。我们采用反证法来完成相关的证明。
    
    取$ \varepsilon=\displaystyle\frac{A-B}{2}>0 $。因为$\lim\limits_{x \to a}f(x)=A $,所以存在$\delta_1>0 $,当 $0<\vert x-a\vert <\delta_1 $时,有$\vert{f(x)-A}\vert<\displaystyle\frac{A-B}{2}$,也即$\displaystyle\frac{A+B}{2}<f(x)<\frac{3A-B}{2}(*)$; 
    
    
    同理,由第二个极限可以得出$\displaystyle\frac{3B-A}{2}<f(x)<\displaystyle\frac{A+B}{2}(**)$。从而,若我们取$\delta=\min{(\delta_1,\delta_2)}$,当
    $ 0<\vert x-a\vert <\delta $时,就有$(*)$与$(**)$同时成立。但$f(x)>\displaystyle\frac{A+B}{2}$与$f(x)<\displaystyle\frac{A+B}{2}$显然不可能同时成立,矛盾,从而假设不成立。
    
    同理,我们可以得到$A<B$也不成立。故$A=B$。\qed
    \end{proof} 
    
    \item \FiveStar 保号性
    \begin{property}
        设$\lim\limits_{x \to a}f(x)=A>(<)0$,则存在$ \delta>0 $,当$ 0<\vert x-a\vert <\delta $时,有$f(x)>(<)0$。
    \end{property}
    \begin{proof}
        设$A>0$。取$\varepsilon=\displaystyle\frac{1}{2}A>0$。因为$\lim\limits_{x \to a}f(x)=A$,故存在$ \delta>0 $,当$ 0<\vert x-a\vert <\delta $时,有 $\vert{f(x)-A}\vert<\varepsilon=\displaystyle\frac{A}{2}$。展开可得$\displaystyle\frac{A}{2}<f(x)<\displaystyle\frac{3}{2}A$。从而$f(x)>0$。\qed
    \end{proof}
    \begin{example}
       若函数$ f(x) $ 满足$ f(1)=0, \lim\limits_{x \to 1} \displaystyle\frac{f'(x)}{(x-1)^3}=-2 $,则$ x=1 $为什么点?
    \end{example}
    \begin{solution}
        因为$\lim\limits_{x \to 1}\displaystyle\frac{f'(x)}{(x-1)^3}=-2<0$,故根据保号性,存在$ \delta>0 $,当$ 0<\vert x-a\vert <\delta $时,有$\displaystyle\frac{f'(x)}{(x-1)^3}<0$。于是,当$x \in (1-\delta,1)$时,$f'(x)>0$;当$x \in (1,1+\delta)$时,$f'(x)<0$。故$x=1$为极大值点。
    \end{solution}
\end{enumerate}

    \subsection{极限的存在性质}
    下面介绍几个判定极限\textbf{存在}的性质。
    \begin{property}
    \begin{enumerate}
        \item 数列型
       
        如果$ a_n\leq b_n\leq c_n $且$ \lim\limits_{n \to \infty}a_n=\lim\limits_{n \to \infty}c_n=A $,则$ \lim\limits_{n \to \infty}b_n=A $。
        \item 函数型
        
        如果$ f(x)\leq g(x)\leq h(x) $且$ \lim\limits_{x \to a}f(x)=\lim\limits_{x \to a}h(x)=A $,则$ \lim\limits_{x \to a}g(x)=A $。
    \end{enumerate}
    \end{property}
    \subsubsection*{型一:$n$项和求极限}
    \begin{example}
        求极限:$\lim\limits_{n \to \infty}(\displaystyle\frac{1}{\sqrt{n^2+1}}+\displaystyle\frac{1}{\sqrt{n^2+2}}+ \cdots +\frac{1}{\sqrt{n^2+n}})$
    \end{example}
    \begin{solution}
        以上是非齐次的情形,采取夹逼定理。于是令$b_n=\displaystyle\frac{1}{\sqrt{n^2+1}}+\displaystyle\frac{1}{\sqrt{n^2+2}}+ \cdots +\frac{1}{\sqrt{n^2+n}}$。
        
        容易得到:
        $\displaystyle\frac{n}{\sqrt{n^2+n}}\leq b_n \leq \displaystyle\frac{n}{\sqrt{n^2+1}}$,
        
        因为$\lim\limits_{n \to \infty}\displaystyle\frac{n}{\sqrt{n^2+n}}=\lim\limits_{n\to\infty}\displaystyle\frac{1}{\sqrt{\frac{1}{n}+1}}=1$且$\lim\limits_{n \to \infty}\displaystyle\frac{n}{\sqrt{n^2+1}}=\lim\limits_{n\to\infty}\displaystyle\frac{1}{\sqrt{\frac{1}{n^2}+1}}=1$,故得到$\lim\limits_{n \to \infty}b_n=1$。
    \end{solution}

    \begin{example}
        求极限$ \lim\limits_{n \to \infty}(\displaystyle\frac{1}{n^2+1}+\displaystyle\frac{1}{n^2+2}+\cdots+\displaystyle\frac{n}{n^2+n}) $。
    \end{example}
    \begin{solution}
        令$ b_n=\displaystyle\frac{1}{n^2+1}+\displaystyle\frac{1}{n^2+2}+\cdots+\displaystyle\frac{n}{n^2+n}$,从而$\displaystyle\frac{1}{2}=\displaystyle\frac{n(n+1)}{2(n^2+n)} \leq b_n \leq\displaystyle\frac{n(n+1)}{2(n^2+1)}$。则$\lim\limits_{n \to \infty}$左$=\lim\limits_{n \to \infty}$右$=\displaystyle\frac{1}{2}$。故所求极限为$\displaystyle\frac{1}{2}$。
    \end{solution}

    \begin{example}
        求极限$ \lim\limits_{n \to \infty}(\displaystyle\frac{1}{n+1}+\displaystyle\frac{1}{n+2}+\cdots+\displaystyle\frac{1}{n+n}) $。
    \end{example}
    \begin{solution}
        \begin{align*}
            \lim_{n \to \infty}(\frac{1}{n+1}+\frac{1}{n+2}+\cdots+\frac{1}{n+n})
            &=\lim_{n \to \infty}\sum_{i=1}^n\frac{1}{n+i} \\
            &=\lim_{n \to \infty}\frac{1}{n}\sum_{i=1}^n\frac{n}{n+i}\\
            &=\lim_{n \to \infty}\frac{1}{n}\sum_{i=1}^n\frac{1}{1+\frac{i}{n}}\\
            &= \int_0^1\frac{1}{1+x}\mathrm{d}x=\ln(x+1)\bigg|_{0} ^1=\ln 2
        \end{align*}
    \end{solution}

    \begin{example}
        求极限$ \lim\limits_{n \to \infty}(\displaystyle\frac{n}{n^2+1^2}+\displaystyle\frac{n}{n^2+2^2}+\cdots+\displaystyle\frac{n}{n^2+n^2}) $。
    \end{example}
    \begin{solution}
        \begin{align*}
            \lim_{n \to \infty}(\frac{n}{n^2+1^2}+\frac{n}{n^2+2^2}+\cdots+\frac{n}{n^2+n^2})
            &=\lim_{n \to \infty}\sum_{i=1}^n\frac{n}{n^2+i^2} \\
            &=\lim_{n \to \infty}\frac{1}{n}\sum_{i=1}^n\frac{n^2}{n^2+i^2}\\
            &=\lim_{n \to \infty}\frac{1}{n}\sum_{i=1}^n\frac{1}{1+\frac{i^2}{n^2}}\\
            &= \int_0^1\frac{1}{1+x^2}\mathrm{d}x=\arctan x\bigg|_{0} ^1=\frac{\pi}{4}
        \end{align*}
    \end{solution}
    \begin{note}
        对于$n$个数相加,分子或分母不齐次的情况,用夹逼定理;
        
        对于分子、分母齐次且分母多一次的情况,用定积分定义。
    \end{note}
    我们还有另一个著名的判定数列极限存在性的定理,也即如下性质:
    \begin{property}
        单调有界数列必有极限。
    \end{property}
    这个性质可以分为两类来讨论,一为单调递增有上界,二为单调递减有下界。
    \subsubsection*{型二:极限存在性证明}
    \begin{example}
        已知,$a_1=\sqrt{2}, a_2=\sqrt{2+\sqrt{2}}, a_3=\sqrt{2+\sqrt{2+\sqrt{2}}}, \cdots$

        证明$\lim\limits_{n \to \infty}a_n$存在,并求之。
    \end{example}
    \begin{solution}
        显然,$ \{a_n\} $单调递增,现在我们证明$ a_n \leq 2$,采用数学归纳法:

        首先,$ a_1=\sqrt{2}<2 $。假设$a_k\leq 2$,则$a_{k+1} =\sqrt{2+a_k}\leq \sqrt{2+2}=2$。因此 $a_n \leq 2$。从而由单调有界定理知$\lim\limits_{n \to \infty}a_n$存在。下面求出这个极限:

        设极限为$A$,由$a_{n+1}=\sqrt{2+a_n}$,令$ n \to \infty $,两边平方有$ A^2=A+2 $,解得$ A=2 $或 $ A=-1 $。由于$a_n>a_1=\sqrt{2}$,故$ A=-1 $舍去。从而极限为$ 2 $。
    \end{solution}

    \begin{example}
        已知$ a_1=2, a_{n+1}=\displaystyle\frac{1}{2}(a_n+\displaystyle\frac{1}{a_n})$。证明:$\lim\limits_{n \to \infty}a_n$存在。
    \end{example}
    \begin{solution}
        先证明$ a_n>0 $:
        由于$a_1>0$,故设$a_k>0$。则$a_{k+1}=\displaystyle\frac{1}{2}(a_k+\displaystyle\frac{1}{a_{k+1}})>0$。于是根据均值不等式可以得到$a_{n+1} \geq 1$。而$a_{n+1}-a_n=\displaystyle\frac{1}{2}(a_n+\displaystyle\frac{1}{a_{n+1}})-a_n=\displaystyle\frac{1}{2}(\displaystyle\frac{1}{a_n}-a_{n+1})$。由于$a_{n} \geq 1$,故$\displaystyle\frac{1}{a_n} \leq a_n$。从而,$a_{n+1}-a_n \leq 0$。于是数列$ \{a_n\} $单调递减,又有$a_n \geq 1$,从而得到$\lim\limits_{n \to \infty}a_n$存在。
    \end{solution}

    \subsection{无穷小的性质}
    \subsubsection*{(一)一般性质}
    \begin{enumerate}
        \item 若$ \alpha \to 0 $且$ \beta \to 0 $,则:
        \[ \left\{
            \begin{array}{rl}
                & \alpha \pm \beta  \to 0,\\
                & k \alpha  \to 0,\\
                & \alpha \beta  \to 0.
            \end{array} \right. \]
        \item 若$ \vert \alpha \vert \leq M, \beta \to 0$,则$\alpha \beta \to 0$。
        \item $\alpha \to 0, \lim\limits f(x)=A \Leftrightarrow f(x)=A+\alpha$。
    \end{enumerate}
    \subsubsection*{(二)等价性质}
    \begin{enumerate}
        \item \begin{enumerate}
            \item $ \alpha \sim \alpha $(自反性);
            \item $ \alpha \sim \beta \Rightarrow \beta \sim \alpha $(对称性);
            \item $ \alpha \sim \beta , \beta \sim \gamma \Rightarrow \alpha \sim \gamma $(传递性)。
        \end{enumerate}
        \item $ \alpha \sim \alpha_1, \beta \sim \beta_1, \lim \displaystyle\frac{\beta_1}{\alpha_1} =A$,则$\lim \displaystyle\frac{\beta}{\alpha} =A$。
        \item \textbf{当$x \to 0$时:} \begin{enumerate}
            \item $x \sim \sin x \sim \tan x \sim \arcsin x \sim \arctan x \sim \mathrm{e}^x -1 \sim \ln{(x+1)}$;
            \item $1- \cos x \sim \displaystyle\frac{1}{2} x^2$;
            \item $(1+x)^a-1 \sim ax$。
        \end{enumerate}
    \end{enumerate}
    
    \section{两个重要极限}
    \subsection{准备工作}
    我们先证明如下结论:

    当$ 0<x<\displaystyle\frac{\pi}{2} $时,有
    \begin{align*}
        \sin x<x<\tan x
    \end{align*}
        \begin{proof}
        如图\ref{dwy}所示,在单位圆中,当$ 0<x<\displaystyle\frac{\pi}{2}$时,有
        $S_{\bigtriangleup AOB}=\displaystyle\frac{1}{2}r^2 \sin x=\displaystyle\frac{1}{2}\sin x=S_1$, 
        $S_{\text{扇形} AOB}=\displaystyle\frac{1}{2}x=S_2$, 
        $S_{\mathrm{Rt} \bigtriangleup AOC}=\displaystyle\frac{1}{2}AC=\displaystyle\frac{1}{2}\tan x=S_3$。
        
        显然有
        \begin{align*}
            S_3>S_2>S_1
        \end{align*}
        从而,
        \begin{align*}
            \frac{1}{2}\tan x>\frac{1}{2}x>\frac{1}{2}\sin x
        \end{align*}
        于是,$\sin x<x<\tan x$证明完毕。 \qed
        \begin{figure}[htbp]
            \centering
            \includegraphics[width=0.7\textwidth]{3.eps}
            \caption{单位圆}
            \label{dwy}
          \end{figure}
        \end{proof}
        \begin{note}
            这里给出的证明并不够严谨,但更严密的证明需要引入幂级数对$\sin x$进行重新定义,这里不再阐述。
        \end{note}
        
        \subsection{两个重要极限式}
        \begin{enumerate}
            \item $\lim\limits_{\Delta \to 0} \frac{\sin \Delta}{\Delta}=1$
            \item $\lim\limits_{\Delta \to 0} (1+\Delta)^{\frac{1}{\Delta}}=\mathrm{e}$
        \end{enumerate}
        \begin{note}
            这里的$\Delta$表示一切具有趋于零状态的变量与表达式,需要当作一个整体来进行处理。
        \end{note}

        \subsubsection*{型三:不定型}
        所谓不定型,就是指含有“无穷”与0的极限求解。包含$ 
        \frac{0}{0} $型,$ 1^{\infty} $型,$ 
        \frac{\infty}{\infty} $型,$ 
        \infty \times \infty $型,$ 
        \infty - \infty $型,$ 
        \infty^{0} $型与$0^{0} $型。

        下面分类进行讲解。
        \begin{enumerate}
            \item $ \displaystyle\frac{0}{0} $型
            
            \begin{enumerate}
                \item 习惯:    
                    \[ \left\{
                        \begin{array}{rl}
                            & u(x)^{v(x)}  \Rightarrow \mathrm{e}^{v(x)\ln u(x)},\\
                            & \ln {(\quad )}   \Rightarrow \ln{(1+\Delta)} \sim \Delta(\Delta \to 0),\\
                            & (\quad)-1  \Rightarrow \left\{ \begin{array}{rl} & \mathrm{e}^{\Delta}-1  \sim \Delta;\\
                                & (1+\Delta)^a-1  \sim a\Delta.
                            \end{array} \right. (\Delta \to 0)\\
                            & x-\ln (1+x) \Rightarrow \text{二阶无穷小},\\
                            & x,\sin x,\tan x, \arcsin x, \arctan x \Rightarrow \text{任意两个之差为三阶无穷小}。
                        \end{array} \right. \]
                \item 注意:例如,$\lim\limits_{x \to 0} \displaystyle\frac{x-\sin x}{x^3} \neq \lim\limits_{x \to 0} \displaystyle\frac{x-x}{x^3}=0$
            \end{enumerate}

            \begin{example}
              求极限: $ \lim\limits_{x\to 0} \displaystyle\frac{\tan x-\sin x}{x^3}$。
            \end{example}
            
            \begin{solution}
                \begin{align*}
                   \text{原式}
                   &=\lim_{x \to 0}\frac{\tan(1-\cos x)}{x^3}\\
                    &=\lim_{x \to 0}\frac{x(\frac{1}{2}x^2)}{x^3}=\frac{1}{2}
                \end{align*}
            \end{solution}

            \begin{example}
                求极限:$\lim\limits_{x \to 0} \displaystyle\frac{(1+x^2)^{\sin x}-1}{x^2 \ln (1+2x)}$。
            \end{example}

            \begin{solution}
                \begin{align*}
                \text{原式}
                &=\lim_{x \to 0}\frac{\mathrm{e}^{\sin x \ln (1+x^2)}-1}{2x^3}\\
                 &=\lim_{x \to 0}\frac{\mathrm{e}^{x^3}-1}{2x^3}=\frac{1}{2}
             \end{align*}
            \end{solution}

            \begin{example}
                求极限:$\lim\limits_{x \to 0} \displaystyle\frac{(\frac{1+\cos x}{2})^x-1}{x^3}$。
            \end{example}

            \begin{solution}
            \begin{align*}
                \text{原式}
                &=\lim_{x \to 0}\frac{\mathrm{e}^{x \ln{\frac{1+\cos x}{2}}}-1}{x^3}\\
                 &=\lim_{x \to 0}\frac{\ln{\frac{1+\cos x}{2}}}{x^2}\\
                 &=\lim_{x \to 0} \frac{\ln{(1+\frac{\cos x-1}{2})}}{x^2}\\
                 &=\lim_{x \to 0}\frac{\frac{\cos x-1}{2}}{x^2}\\
                 &=\lim_{x \to 0}\frac{\frac{1}{2} (-\frac{1}{2}x^2)}{x^2}=-\frac{1}{4}
            \end{align*}
            \end{solution}

            \item $ 1^{\infty} $型:
            

            主要有两种方法:一是凑出$(1+\Delta)^{\frac{1}{\Delta}}$的形式,二是恒等变形。

            \begin{example}
                求极限:$\lim\limits_{ x \to 0}(1-x\sin x)^{\frac{1}{x - \ln (1+x)}}$。
            \end{example}
            
            \begin{solution}
                \begin{align*}
                    \text{原式}
                    &=\lim_{ x\to 0}(1+(-x \sin x))^{-\frac{1}{x \sin x} \cdot -\frac{x \sin x}{x-\ln (1+x)}}\\
                    &= \mathrm{e}^{-\lim\limits_{x \to 0}\frac{x \sin x}{x-\ln(1+x)}}\\
                    &= \mathrm{e}^{-\lim\limits_{x \to 0}\frac{x^2}{x -\ln(1+x)}}\\
                    &= \mathrm{e}^{-\lim\limits_{x \to 0}\frac{2x}{\frac{x}{1+x}}}\\
                    &= \mathrm{e}^{-\lim\limits_{x \to 0}2(x+1)}=\mathrm{e}^{-2}
                \end{align*}
            \end{solution}

            \begin{example}
                求极限:$\lim\limits_{x \to \infty}(\cos \frac{1}{x})^{x^2}$。
            \end{example}
            
            \begin{solution}
                \begin{align*}
                    \text{原式}
                    &= \lim_{x \to \infty} (1+\cos \frac{1}{x}-1)^{\frac{1}{\cos{\frac{1}{x}-1}}x^2(\cos\frac{1}{x}-1)}\\
                    &=\mathrm{e}^{\lim\limits_{x\to \infty}x^2(\cos \frac{1}{x}-1)}\\
                    &=\mathrm{e}^{\lim\limits_{x \to \infty}\frac{\cos\frac{1}{x}-1}{\frac{1}{x^2}}}\\
                    &\xlongequal{t=\frac{1}{x}}\mathrm{e}^{\lim\limits_{t \to 0}\frac{\cos t-1}{t^2}}\\
                    &=\mathrm{e}^{-\frac{1}{2}}
                \end{align*}
            \end{solution}

            \begin{example}
                求极限:$\lim\limits_{x \rightarrow 0}\left(\displaystyle\frac{1+x}{1+\sin x}\right)^{\frac{1}{x^{3}}}$
            \end{example}

            \begin{solution}
                \begin{align*}
                    \text{原式}
                    &=\lim_{ x \to 0}\left[\left(1+\frac{x-\sin x}{1+\sin x}\right)^{\frac{1+ \sin x}{x-\sin x}}\right]^{\frac{x-\sin x}{1+\sin x}\frac{1}{x^3}}\\
                    &=\mathrm{e}^{\lim\limits_{x \to 0}\frac{1}{1+\sin x}\frac{x -\sin x}{x^3}}\\
                    &=\mathrm{e}^{\lim\limits_{x \to 0}\frac{1 -\cos x}{3x^2}}\\
                    &=\mathrm{e}^{\frac{1}{6}}
                \end{align*}
            \end{solution}

            \item $\infty - \infty$ 型:
            \begin{example}
                求极限:$\lim\limits_{x \to 0}(\displaystyle\frac{1}{x^2}-\displaystyle\frac{1}{(\tan x)^2})$
            \end{example}

            \begin{solution}
                \begin{align*}
                    \text{原式}
                    &=\lim_{x \to 0}(\frac{(\tan x)^2-x^2}{(\tan x)^2 x^2})\\
                    &=\lim_{x \to 0}\frac{(\tan x)^2 -x^2}{x^4}\\
                    &=\lim_{x \to 0}\frac{\tan x +x}{x} \times \frac{\tan x -x}{x^3}\\
                    &=2\lim_{x \to 0}\frac{(\sec x)^2 -1}{3x^2}\\
                    &=\frac{2}{3}\lim_{x \to 0}\frac{(\tan x)^2}{x^2}=\frac{2}{3}
                \end{align*}
            \end{solution}

            \begin{example}
                求极限:$\lim\limits_{x \to \infty}(\sqrt{x^2-4x+8}-x)$
            \end{example}

            \begin{solution}
                \begin{align*}
                    \text{原式}
                    &=\lim_{x \to \infty}\frac{-4x+8}{\sqrt{x^2-4x+8}+x}\\
                    &=\lim_{x \to \infty}\frac{-4+\frac{8}{x}}{\sqrt{1-\frac{4}{x}+\frac{8}{x^2}}+1}\\
                    &=2
                \end{align*}
            \end{solution}

            \item $ \displaystyle\frac{\infty}{\infty} $型:
            \begin{enumerate}
                \item                     
                \[ \lim_{x \to \infty} \frac{b_m x^m+b_{m-1}x^{m-1}+\cdots+b_0}{a_n x^n+a_{n-1}x^{n-1}+\cdots+a_0}
                \left\{
                    \begin{array}{rl}
                        &= 0, \quad m<n\\
                        &= \frac{b_m}{a_m},\quad m=n\\
                        &= \infty,\quad m>n
                    \end{array} \right. \]
           

        \begin{example}
            已知:$\lim\limits_{n \to \infty}\displaystyle\frac{n^{2019}}{(n+1)^a-n^a}=k(\neq 0,\neq \infty)$,求$a,k$。
        \end{example}
        
        \begin{solution}
            \begin{align*}
                (n+1)^a &=\mathrm{C}_a^0 n^a+\mathrm{C}_a^1 n^{a-1}+\cdots+\mathrm{C}_a^a\\
                &=n^a+an^{a-1}+A
            \end{align*}
            所以,$(n+1)^a-n^a=an^{a-1}+A$。从而,$a-1=2019 \Rightarrow a=2020$,则$k=\displaystyle\frac{1}{2020}$。
        \end{solution}

        \begin{example}
            求极限:$\lim\limits_{x \to +\infty}\displaystyle\frac{\ln (x^2+3)}{\ln(x^4+3x+1)}$
        \end{example}

        \begin{solution}
            \begin{align*}
                \text{原式}
                &=\lim_{x \to +\infty}\frac{\ln{x^2(1+\frac{3}{x^2})}}{\ln{x^2(1+\frac{3}{x^3}+\frac{1}{x^4})}}\\
                &=\lim_{x \to +\infty}\frac{2\ln x+\ln(1+\frac{3}{x^3})}{4\ln x+\ln(1+\frac{3}{x^3}+\frac{1}{x^4})}=\frac{1}{2}
            \end{align*}
        \end{solution}

        \item 洛必达法则
        \begin{example}
            $\lim\limits_{x \to +\infty}\displaystyle\frac{x^3}{\mathrm{e}^x}=\lim\limits_{x \to +\infty}\displaystyle\frac{6}{\mathrm{e}^x}=0$
        \end{example}

        \begin{example}
            $\lim\limits_{x \to +\infty}\displaystyle\frac{\ln x}{\sqrt{x}}=\lim\limits_{x \to +\infty}\displaystyle\frac{\frac{1}{x}}{\frac{1}{2\sqrt{x}}}=\lim\limits_{x \to +\infty}\displaystyle\frac{2}{\sqrt{x}}=0$
        \end{example}
        \end{enumerate}

        \item $0 \times \infty$型

        可以化成如下两种情形:
        \[ \left\{
                        \begin{array}{rl}
                             \frac{0}{\frac{1}{\infty}} &\text{即} \frac{0}{0}\text{型}\\
                             \frac{\infty}{\frac{1}{0}} &\text{即} \frac{\infty}{\infty}\text{型}
                        \end{array} \right. \]

            
    

                        
        \begin{example}
            求极限:$\lim\limits_{x \to +\infty}(2x+1)^2 \sin{\frac{1}{x^2}}$。
        \end{example}  
        
        \begin{solution}
            \begin{align*}
                \text{原式}
                &=\lim_{x \to +\infty}\frac{(2x+1)^2}{x^2} \times \frac{\sin{\frac{1}{x^2}}}{\frac{1}{x^2}}=4
            \end{align*}
        \end{solution}

        \begin{example}
            求极限:$\lim\limits_{x \to +\infty}(x-x^2\ln(1+\displaystyle\frac{1}{x}))$。
        \end{example}

        \begin{solution}
            \begin{align*}
                \text{原式}
                &=\lim_{x \to +\infty}x^2(\frac{1}{x}-\ln(1+\frac{1}{x}))\\
                &=\lim_{x \to +\infty}\frac{\frac{1}{x}-\ln(1+\frac{1}{x})}{\frac{1}{x^2}}\\
                &\xlongequal{t=\frac{1}{x}}\lim_{t \to 0}\frac{t-\ln(1+t)}{t^2}\\
                &=\lim_{t \to 0}\frac{1-\frac{1}{t+1}}{2t}\\
                &=\lim_{t \to 0}\frac{\frac{t}{t+1}}{2t}=\frac{1}{2}
            \end{align*}
        \end{solution}

        \item $0^0$型和$\infty ^{0}$型

        这两种都可以化成$\mathrm{e}^{\ln}$的形式。

        \begin{example}
            $\lim\limits_{x \to 0^{+}}x^x=\lim\limits_{x \to 0^{+}}\mathrm{e}^{x \ln x}=\lim\limits_{x \to 0^{+}}\mathrm{e}^{\frac{\ln x}{1/x}}=\lim\limits_{x \to 0^{+}}\mathrm{e}^{\frac{1/x}{-1/x^2}}=0$
        \end{example}
        \end{enumerate}

        \subsubsection*{型四:左右极限}
        \FiveStar $a^{\displaystyle\frac{?}{x-b}}$或$a^{\displaystyle\frac{?}{b-x}}$当$x \to b$时一定分左右。

        \begin{example}
            已知$ f(x)=\mathrm{e}^{\frac{x}{x-2}}$,求极限$\lim\limits_{x \to 2}f(x)$。
        \end{example}

        \begin{solution}
            因为 $f(2-0)=0 , f(2+0)=+\infty$,故该极限不存在。
        \end{solution}

        \begin{example}
            $f(x)=\displaystyle\frac{1-2^{\frac{1}{x-1}}}{1+2^{\frac{1}{x-1}}}$,求$\lim\limits_{x \to 1}f(x)$。
        \end{example}

        \begin{solution}
            因为$f(1-0)=\displaystyle\frac{1}{1}=1, f(1+0)=\displaystyle\frac{1-\infty}{1+\infty}=-1$,故$\lim\limits_{x \to 1}f(x)$不存在。
        \end{solution}
        这个函数的图像如下图所示:
        \begin{figure}[htbp]
            \centering
            \includegraphics[width=0.7\textwidth]{4.eps}
            \caption{$f(x)$的图像}
          \end{figure}

          \section{连续与间断}
          \subsection{相关定义}
          \begin{enumerate}
              \item 连续
              \begin{enumerate}
                  \item 一点连续:
                  若 $\lim\limits_{x \to a}f(x)=f(a)$ 或$f(a-0)=f(a+0)=f(a)$,则称$f(x)$在$x = a$连续。
                  \item $f(x)$在$[a,b]$上连续:若 $f(x)$在$(a,b)$内连续,且$f(a)=f(a+0),f(b)=f(b-0)$。则称$f(x)$在$[a,b]$上连续,记为$f(x)\in C[a,b]$。
              \end{enumerate}
              \item 间断: 若 $\lim\limits_{x \to a}f(x)\neq f(a)$
              
              分类:
              \begin{enumerate}
                  \item 第一类:$f(a+0),f(a-0)$均存在。
                  
                  若 $f(a-0)=f(a+0)\neq f(a)$,则称为可去间断点;

                  若 $f(a-0)\neq f(a+0)$,则称为跳跃间断点。
                  \item 第二类:$f(a-0),f(a+0)$至少一个不存在。
              \end{enumerate}
          \end{enumerate}

          \subsubsection*{型六:间断点及分类}
          \begin{example}
              求函数$ f(x) =\displaystyle\frac{x^2-3x+2}{x^2-1}e^{\frac{1}{x}}$的间断点。
          \end{example}

          \begin{solution}
              容易看出,$ x=0, x=\pm 1$为间断点。又因为 $f(0-0)=0$ 且 $f(0+0)=-\infty$,故 $x=0$ 为第二类间断点。而 $\lim\limits_{x\to -1}f(x)=\infty$,$\lim\limits_{x \to 1} = \lim\limits_{x \to 1}\frac{x-2}{x+1}e^{-\frac{1}{x}}=-\frac{1}{2}e$,所以可以得到 $x=-1$ 为第二类间断点,而 $x=1$ 为可去间断点。
          \end{solution}
   
          \begin{example}
            求函数$ f(x) =\displaystyle\frac{x e\frac{1}{x-1}}{1-e^{\frac{x}{x-1}}}$的间断点。
        \end{example}

        \begin{solution}
            容易看出,$ x=0, x=1$为间断点。因为 $x \to 0$ 时,  $1-e^{\frac{x}{x-1}} \sim \displaystyle\frac{x}{1-x}$,故 $\lim\limits_{x \to 0}f(x) = \lim\limits_{x \to 0}e^{\frac{1}{x-1}}\times \displaystyle\frac{x}{1-e^{\frac{x}{x-1}}}=\lim\limits_{x\to 0}e^{-1}\times\displaystyle\frac{x}{\frac{x}{1-x}}=e^{-1}$,从而 $x=0$ 为可去间断点。而 $f(1-0)=0 \neq f(1+0)=-\lim\limits_{x\to 1^+}e^{\frac{1-x}{x-1}}=-e^{-1}$,从而 $x=1$ 为跳跃间断点。
        \end{solution}

        \subsection{函数在区间上连续}
        \begin{enumerate}
            \item 最值
            
            若 $f(x)$ 在 $[a,b]$ 上连续,则 $f(x)$ 在 $[a,b]$ 内存在最小值 $m$ 与最大值$M$;
            \item 有界
            
            若 $f(x)$ 在 $[a,b]$ 上连续,则存在 $k>0$ ,使得 $|f(x)|\leq k$;
        \begin{figure}[htbp]
            \centering
            \includegraphics[width=0.3\textwidth]{5.png}
            \caption{$f(x)$的图像}
          \end{figure}

          \item 零点定理
          
          若 $f(x) \in [a,b]$ 且 $f(a)f(b)<0$ ,则 $\exists c\in [a,b]$,使得 $f(c)=0$。
          \item 介值定理
           
          $f(x) \in C[a,b] $,则 $\forall \eta \in [m,M],\exists \xi \in [a,b]$,使得 $f(\xi)=\eta $。(位于最小值 $m$ 和最大值 $M$ 之间的任何值皆可被 $f(x)$ 取到)。
        \end{enumerate}

        \begin{example}
            证明:方程 $x^5-4x+1=0$ 至少有一个正根。
        \end{example}

        \begin{solution}
            令 $f(x)=x^5-4x+1$。由于 $f(0)=1>0,f(1)=1-4+1<0$,故根据零点定理,存在 $x_0 \in (0,1)$,使得 $f(x)=0$,也即方程 $x^5-4x+1=0$ 至少有一个正根。\qed
        \end{solution}

        \begin{example}
            $f(x)\in C[a,b],p,q>0$,且$p+q=1$。

            证明:存在 $\xi \in [a,b]$,使得 $f(\xi)=pf(a)+qf(b)$。
        \end{example}

        \begin{solution}
            由于 $f(x)\in C[a,b]$ ,故 $f(x)$ 在  $[a,b]$ 上存在最小值 $m$ 和最大值 $M$。而 $pf(a)+qf(b)\leq pM+qM=M$,且 $pf(a)+qf(b)\geq  pm+qm=m$,从而 $pf(a)+qf(b) \in [m,M]$。根据介值定理,存在 $\xi \in [a,b]$, $f(\xi)=pf(a)+qf(b)$。 \qed
        \end{solution}

        \begin{example}
            $f(x) \in C[0,2],f(0)+2f(1)+3f(2)=6$。证明:$\exists c \in [0,2]$,使得 $f(c)=1$。
        \end{example}

        \begin{solution}
            由于 $f(x)\in C[0,2]$ ,故 $f(x)$ 在  $[0,2]$ 上存在最小值 $m$ 和最大值 $M$。于是 $6m\leq f(0)+2f(1)+3f(2) \leq 6M$,故 $m \leq 1 \leq M$。根据介值定理,存在 $c\in[0,2]$, $f(c)=1$。\qed
        \end{solution}

